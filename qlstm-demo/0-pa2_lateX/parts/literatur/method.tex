\section{Methodik der Literaturanalyse}

Die gesamte Literaturanalyse wurde systematisch nach der Methode \cite{Webster2002} durchgeführt. Dabei wurden die als relevant identifizierten Konzepte erfasst\footcite[Vgl.][S. 15]{Webster2002}:

\textit{\ac{LSTM}},
\textit{\ac{QLSTM}},
\textit{\ac{QML}},
\textit{\ac{VQC}} und
\textit{Quantum Data Encoding}. 

Die erstellte Konzeptmatrix in~\ref{konzeptmatrix} zeigt die Zusammenhänge der Konzepte untereinander.
Um passende Literatur zu finden, wurde eine Liste an Suchbegriffen erstellt. Diese ist in~\ref{search} zu finden. Die Suchbegriffe wurden in den folgenden Datenbanken gesucht: Google Scholar\footcite{https://scholar.google.com/}, IEEE Xplore\footcite{https://ieeexplore.ieee.org/}, ScienceDirect\footcite{https://www.sciencedirect.com/} und arXiv\footcite{https://arxiv.org/}.
Die jeweils ersten 10 Suchergebnisse wurden auf Relevanz geprüft, und die passenden Artikel wurden in der Konzeptmatrix aufgenommen.
Von relevanten Artikeln wurden außerdem Referenzen und Zitationen mit ConnectedPapers\footcite{https://www.connectedpapers.com/} visualisiert und ebenfalls geprüft.

Im nächsten Teil der Literaturanalyse werden nun die einzelnen Konzepte und ihre Zusammenhänge erläutert. Eine Analyse der Vergleichsobjekte findet in Kapitel~\ref{vergleichsobjekte} statt.
