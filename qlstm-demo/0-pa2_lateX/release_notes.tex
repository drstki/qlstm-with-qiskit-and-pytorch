\anhang{Release Notes}
\anhangteil{Änderungen in Version 1.1}\label{anhang:ReleaseNotes11}
In Version 1.1 sind einige Rückmeldungen, die nach der Einführungsvorlesung am 6.2.2015 oder nach Veröffentlichung der Vorlage in Moodle eingegangen sind, berücksichtigt worden. Korrekturen sind mit \enquote{(Fix)} gekennzeichnet. 

\begin{itemize}
\item \verb|latex-vorlage.tex|
\begin{itemize}
\item (Fix) Abkürzungsverzeichnis wird vor Abbildungsverzeichnis platziert
\item (Fix) Abbildungs- und Tabellenverzeichnis in Inhaltsverzeichnis aufgenommen
\item (Fix) Quellenverzeichnis wird nun ohne Kapitelnummer dargestellt

\item eingebundene Dateien in Unterverzeichnissen \verb|includes| bzw.\ \verb|graphics|
\item Beispiel-Anhang (Datei \verb|anhang.tex|) mit Erklärungen wurde eingebunden 
\end{itemize}

\item \verb|_dhbw_praeambel.tex|
\begin{itemize}
\item (Fix) das Paket hyperref wird nach biblatex eingebunden, um ein Problem mit der Verlinkung der Fußnoten im PDF zu beheben
\item (Fix) Fußnoten  gemäß der Richtlinien fortlaufend nummeriert und nicht pro Kapitel
\item Einstellungen hinzugefügt, um Anhangsverzeichnis zu ermöglichen
\item bessere Kompatibilität zwischen KOMA-Script (scrreprt) und anderen Paketen mittels scrhack
\end{itemize}

\item \verb|_dhbw_biblatex-config.tex|
\begin{itemize}
\item (Fix) keine Abschnittsnummern für einzelne Verzeichnisse im Quellenverzeichnis
\end{itemize}

\item \verb|abbildungen_und_tabellen.tex|
\begin{itemize}
\item Erklärung, wie eine Fußnote/ein Zitat bei einer Abbildung zu erstellen ist
\end{itemize}

\item \verb|abkuerzungen.tex|
\begin{itemize}
\item Abkürzungsverzeichnis wird im Inhaltsverzeichnis aufgeführt
\end{itemize}

\item \verb|abstract.tex|, \verb|anhang.tex|, \verb|einleitung.tex| 
\begin{itemize}
\item Erklärungen im Text ergänzt
\end{itemize}

\item \verb|deckblatt.tex|
\begin{itemize}
\item Meta-Daten (Autor, Titel) für die generierte PDF-Datei lassen sich nun festlegen
\end{itemize}

\end{itemize}


\anhangteil{Änderungen in Version 1.2}\label{anhang:ReleaseNotes12}
Über das Forum in Moodle sind einige Rückmeldungen eingegangen -- vielen Dank an alle, die dazu beigetragen haben. In der Version 1.2 wurden folgende Änderungen vorgenommen, wobei Korrekturen wieder mit \enquote{(Fix)} gekennzeichnet sind: 

\begin{itemize}
\item \verb|latex-vorlage.tex| (Hauptdokument)
\begin{itemize}
\item (Fix) Zeile 19: Seitenzahlen zu Beginn mit römischen \emph{Groß}buchstaben nummeriert
\end{itemize}

\item \verb|_dhbw_praeambel.tex|
\begin{itemize}
\item Zeile 39/40: Unterstützung für \enquote{ebenda} 
\item Zeile 46--68: zweite Gliederungsebene für Anhänge ermöglicht
\item (Fix) Zeile 70--73: Abbildungen und Tabellen: Zähler fortlaufend, kein Rücksetzen zu Kapitelbeginn (Paket \verb|chngcntr| anstelle von Paket \verb|remreset|)
\end{itemize}

\item \verb|_dhbw_biblatex-config.tex|
\begin{itemize}
\item (Fix) bei Quellen mit Herausgeber, aber ohne Autor wird der Name des Herausgebers im Verzeichnis fett gedruckt
\item Unterstützung für \enquote{ebenda} 
\end{itemize}

\item \verb|abkuerzungen.tex|
\begin{itemize}
\item Bemerkungen zur fortgeschrittenen Nutzung des \verb|acronym|-Pakets eingefügt 
\end{itemize}

\item \verb|einleitung.tex|
\begin{itemize}
\item Abschnitt 1.3 zu Einstellungen ergänzt
\item Abschnitt 1.5 zu Fehlerbehebungen eingefügt 
\end{itemize}

\item \verb|text-mit-zitaten.tex|
\begin{itemize}
\item Abschnitt 3.1 eingefügt, Erläuterungen zum Zitieren mit \enquote{vgl.} und \enquote{ebenda}. 
\item Abschnitt 3.2: Beispiele ergänzt
\item Hinweis zu Jahreszahlen bei Online-Quellen
\end{itemize}

\item \verb|anhang.tex|
\begin{itemize}
\item Erläuterungen zur zweiten Gliederungsebene
\end{itemize}

\item \verb|literatur-datenbank.bib|
\begin{itemize}
\item weitere Beispiele für Quellen
\end{itemize}

\end{itemize}

\anhangteil{Änderungen in Version 1.3}\label{anhang:ReleaseNotes13}
Durch die ab 1/2016 geltenden Änderungen der Zitierrichtlinien des Studiengangs waren einige kleinere Anpassungen der Vorlage erforderlich, die nachfolgend beschrieben sind. Bei dieser Gelegenheit ebenfalls erfolgte Korrekturen sind wieder mit \enquote{(Fix)} gekennzeichnet:

\begin{itemize}
\item \verb|latex-vorlage.tex| (Hauptdokument)
\begin{itemize}
\item Hinweis auf Option doppelseitiger Druck entfernt
\item Schriftgröße der Kapitelüberschriften verkleinert
\item (Fix) Kopf- und Fußzeilen werden nun korrekt angezeigt für erste Seite eines Kapitels und auch  Quellenverzeichnisse
\end{itemize}

\item \verb|_dhbw_praeambel.tex|
\begin{itemize}
\item Angabe des unteren Rands für Seitenzahl, da diese nun unten rechts steht
\item Unterstützung für \enquote{ebenda} entfernt
\item (Fix) Präfixe wie \enquote{von} im Namen eines Autors werden berücksichtigt
\item Anpassung der Abstände bei Kapitelüberschriften
\item Kopf- und Fußzeile für Verzeichnisse nun in \verb|_dhbw_kopfzeilen.tex| definiert 
\end{itemize}


\item \verb|deckblatt.tex|
\begin{itemize}
\item Schriftgröße des Titels vergrößert
\item Befehl \verb|\typMeinerArbeit| eingeführt, um Typ auszuwählen
\item Festlegung des Themas (für ehrenwörtliche Erklärung) mit Befehl \verb|\themaMeinerArbeit|
\item Darstellung der Angabe des Betreuers in der Ausbildungsstätte angepasst
\item Formulierung des Sperrvermerks angepasst  
\end{itemize}

\item \verb|_dhbw_erklaerung.tex|
\begin{itemize}
\item Formulierung angepasst an geänderte Prüfungsordnung
\item Typ und Thema der Arbeit werden automatisch eingefügt
\end{itemize}

\item \verb|_dhbw_kopfzeilen.tex|
\begin{itemize}
\item Seitennummern stehen jetzt unten rechts
\item (Fix) Kopf- und Fußzeile werden nun korrekt angezeigt in Verzeichnissen und dem Anhang
\end{itemize}

\item \verb|_dhbw_biblatex-config.tex|
\begin{itemize}
\item Anpassung des Zitierstils auf die ab 1/2016 geltenden Regelungen  
\item Vorkehrungen für Eindeutigkeit (Hinzufügen abgekürzter oder nötigenfalls ausgeschriebener Vorname) bei Übereinstimmung von Name und Jahreszahl 
\end{itemize}

\item \verb|einleitung.tex|
\begin{itemize}
\item Abschnitt 1.3 zu Einstellungen grundlegend überarbeitet
\item Abschnitt 1.5.2 zur Kontrolle der Seitenränder eingefügt 
\end{itemize}

\item \verb|text-mit-zitaten.tex|
\begin{itemize}
\item Abschnitt 3.1: Hinweise zu \enquote{ebenda} entfernt
\item Abschnitt 3.2: Beispiele zur Eindeutigkeit des Zitats ergänzt
\item Abschnitt 3.3: Hinweise für E-Journals/E-Books ergänzt 
\end{itemize}

\item \verb|anhang.tex|
\begin{itemize}
\item (Fix) Befehl \verb|\spezialkopfzeile| aufgenommen, damit in Kopfzeile das Wort \enquote{Anhang} angezeigt wird 
\item diese Release Notes wurden in eine eigene Datei verschoben
\end{itemize}

\item \verb|release_notes.tex|
\begin{itemize}
\item s.o.
\end{itemize}


\item \verb|literatur-datenbank.bib|
\begin{itemize}
\item weitere Beispiele für Quellen
\end{itemize}
\end{itemize}

\anhangteil{Änderungen in Version 1.4}\label{anhang:ReleaseNotes14}
Durch nicht abwärtskompatible Änderungen beim Versionswechsel von Biblatex 3.2 zu 3.3 sind einige Änderungen notwendig geworden.\footnote{Diese basieren auf Vorschlägen von Yannik Ehlert -- vielen Dank dafür!}
Die vorliegende Version 1.4 wurde erfolgreich mit MikTeX gestestet (portable Version 2.9.6361 vom 3.6.2017, unter Verwendung von Biblatex 3.7).

\begin{itemize}
\item \verb|_dhbw_biblatex-config.tex|
\begin{itemize}
\item Anpassung der \verb|\usebibmacro|-Befehle
\end{itemize}

\item \verb|_dhbw_authoryear.bbx|
\begin{itemize}
\item  Änderung von \verb|\printdateextralabel| zu \verb|\printlabeldateextra|
\end{itemize}
\end{itemize}

\anhangteil{Änderungen in Version 1.5}\label{anhang:ReleaseNotes15}
Für den Test dieser Version auf einem Windows-System wurde wieder die portable Version von MiKTeX (2.9.6521 vom 10.11.2017) verwendet.\footnote{\url{http://miktex.org/portable}} Da in diesem Paket leider die Versionen von Biblatex (3.10) und Biber (2.7) inkompatibel sind, ist es erforderlich, die Datei \verb|biber.exe| im Verzeichnis \verb|texmfs\install\miktex\bin\| durch die aktuelle Version 2.10 vom 20.12.2017\footnote{\url{https://sourceforge.net/projects/biblatex-biber/files/biblatex-biber/current/binaries/Windows/}} zu ersetzen. Im Editor TeXworks verwendet man dann zum Übersetzen des \LaTeX-Sourcecodes Typeset/pdfLaTeX bzw.\ Typeset/Biber.

Korrekturen sind wieder mit \enquote{(Fix)} gekennzeichnet.

\begin{itemize}
\item \verb|latex-vorlage.tex| (Hauptdokument)
\begin{itemize}
\item Nach der Änderung der Zitierrichtlinien gibt es nun kein separates Verzeichnis mehr für Internet- und Intranetquellen.
\item Option \verb|notkeyword=ausblenden| bei \verb|\printbibligraphy| sorgt dafür, dass Sekundärliteratur korrekt zitiert wird.
\end{itemize}

\item \verb|_dhbw_praembel.tex|
\begin{itemize}
\item (Fix) Die Bezeichnung geschachtelter Anhänge wurde auf das in den Zitierrichtlinien geforderte Format \enquote{Anhang 2/1} angepasst (Befehl \verb|\anhangteil|).
\end{itemize}

\item \verb|einleitung.tex|
\begin{itemize}
\item Hinweis zum Ausblenden der farbigen Links im PDF hinzugefügt
\end{itemize}

\item \verb|text-mit-zitaten.tex|
\begin{itemize}
\item Abschnitt 3.4 aktualisiert nach Wegfall des separaten Verzeichnisses für Internet- und Intranetquellen
\item Abschnitt zum Zitieren von Sekundärliteratur hinzugefügt
\end{itemize}

\end{itemize}


\anhangteil{Änderungen in Version 1.6}\label{anhang:ReleaseNotes16}
Diese Version wurde auf einem Windows-System erfolgreich mit der portablen Version von MiKTeX (2.9.6621 vom 18.02.2018) getestet.\footnote{Vielen Dank an Florian Eichin für seine wertvollen Anmerkungen.}

Korrekturen sind wieder mit \enquote{(Fix)} gekennzeichnet.

\newpage

\begin{itemize}
\item \verb|latex-vorlage.tex| (Hauptdokument)
\begin{itemize}
\item (Fix) An einer Stelle gab es in Version 1.5 (Internetquellen nicht mehr separat) noch ein Überbleibsel von Version 1.4 (Internetquellen separat), dies wurde korrigiert.
\item (Fix) Im Inhaltsverzeichnis war die Verlinkung des Abbildungs- und Tabellenverzeich\-nisses nicht ganz korrekt.
\item Mit den Befehlen \verb|\literaturverzeichnis| bzw.\ \verb|\literaturUndQuellenverzeichnis| kann bequem die Erstellung der Quellenverzeichnisse gesteuert werden, abhängig davon, ob es ein Gesprächsverzeichnis gibt oder nicht.
 
\end{itemize}

\item \verb|_dhbw_praembel.tex|
\begin{itemize}
\item Einrückungen für Abbildungs-, Tabellen- und Anhangverzeichnis angepasst
\item Abkürzungen \enquote{Abb.} und \enquote{Tab.} für Abbildungen bzw.\ Tabellen
\end{itemize}

\item \verb|_dhbw_biblatex-config.tex|
\begin{itemize}
\item Befehle \verb|\literaturverzeichnis| und \verb|\literaturUndGespraechsverzeichnis| definiert
\item Befehl \verb|\footcitePrimaerSekundaer| definiert
\end{itemize}

\item \verb|_dhbw_erklaerung.tex|
\begin{itemize}
\item Eintrag als \enquote{Erklärung} (statt \enquote{Ehrenwörtliche Erklärung}) ins Inhaltsverzeichnis
\end{itemize}

\item \verb|einleitung.tex|
\begin{itemize}
\item Bezeichnung \enquote{Erklärung} statt \enquote{Ehrenwörtliche Erklärung}
\item Erläuterung von \verb|\literaturverzeichnis| und \verb|\literaturUndGespraechsverzeichnis|
\item Hinweis auf Notwendigkeit von Updates bei MikTeX Portable
\end{itemize}

\item \verb|text_mit_zitaten.tex|
\begin{itemize}
\item Erläuterungen zu Befehl \verb|\footcitePrimaerSekundaer| ergänzt
\end{itemize}

\item \verb|anhang.tex|
\begin{itemize}
\item Befehl \verb|\abstaendeanhangverzeichnis| für Anpassung Einrückung ergänzt
\end{itemize}

\item \verb|literatur-datenbank.bib|
\begin{itemize}
\item Eintrag ergänzt
\end{itemize}

\end{itemize}

\anhangteil{Änderungen in Version 1.7}\label{anhang:ReleaseNotes17}
Diese Version wurde auf einem Windows-System erfolgreich mit der portablen Version von MiKTeX (2.9.6942 vom 04.01.2019) getestet.

Korrekturen sind wieder mit \enquote{(Fix)} gekennzeichnet.

\begin{itemize}
\item \verb|_dhbw-authoryear.bbx|
\begin{itemize}
\item Da \verb|labeldate| in Biblatex nicht mehr unterstützt wird, erfolgte eine Umbenennung in 
\verb|labeldateparts|.\footnote{vgl.\ \url{https://github.com/semprag/biblatex-sp-unified/issues/23}}
\end{itemize}

\item \verb|_dhbw_biblatex-config.tex|
\begin{itemize}
\item (Fix) Es wurde das Problem behoben, dass im Literaturverzeichnis bei bestimmten Eintragstypen der Titel in Anführungszeichen steht.\footnote{Danke an Florian Eichin für seinen Hinweis.}
\end{itemize}

\end{itemize}


\anhangteil{Änderungen in Version 1.8}\label{anhang:ReleaseNotes18}
Diese Version wurde auf einem Windows-System erfolgreich mit der portablen Version von MiKTeX (2.9.6942 vom 04.01.2019) getestet.

Die Aktualisierungen in der Vorlage spiegeln zum Einen die Änderungen in den Zitierrichtlinien wieder. Zum Anderen wurden einige studentische Vorschläge aufgegriffen, um die Nutzung der Vorlage zu erleichtern.\footnote{Danke an Bjarne Koll, Tobias Schwarz und Lars Ungerathen für ihre Anregungen.} 

\begin{itemize}

\item \verb|latex_vorlage.tex| (Hauptdokument)
\begin{itemize}
\item Es wird nun davon ausgegangen, dass die zur Vorlage gehörenden Dateien in einem eigenen Verzeichnis (\verb|template|) liegen.
\item Stellenweise wurden Erläuterungen als Kommentare hinzugefügt.
\end{itemize}

\item \verb|_dhbw_biblatex-config.tex|
\begin{itemize}
\item Code, der mehrere Quellenverzeichnisse unterstützt, wurde entfernt.
\item Ein zu großer Abstand nach Zitaten von Sekundärliteratur wurde korrigiert. 
\end{itemize}

\item \verb|_dhbw_erklaerung.bbx|
\begin{itemize}
\item Gemäß der Anforderung in den Zitierrichtlinien wird die Erklärung nicht ins Inhaltsverzeichnis aufgenommen und nicht mit einer Seitenzahl versehen. 
\end{itemize}

\pagebreak
\item \verb|_dhbw_praeambel.bbx|
\begin{itemize}
\item Gemäß der Anforderung in den Zitierrichtlinien werden im Literaturverzeichnis alle Autor/innen eines Werks angegeben.
\end{itemize}

\item \verb|abstract.tex|
\begin{itemize}
\item Hinweis auf \LaTeX-Spickzettel hinzugefügt.
\end{itemize}

\item \verb|deckblatt.tex|
\begin{itemize}
\item Vorname, Name, Titel der Arbeit sind nur zu Beginn einzutragen und werden dann an den entsprechenden Stellen automatisch ergänzt.
\item Hervorhebung, dass Angaben zum Unternehmen sowie den Betreuer/innen zu ergänzen sind. 
\item Wortlaut des Vertraulichkeitsvermerks wurde an die aktuelle Fassung in der Studien- und Prüfungsordnung angepasst. 
\end{itemize}

\item \verb|einleitung.tex|
\begin{itemize}
\item Ein eigenständiges Gesprächsverzeichnis als Teil des Quellenverzeichnisses ist in den Zitierrichtlinien nicht mehr vorgesehen, die entsprechenden Hinweise wurden entfernt.
\item Ein alter Hinweis auf die Darstellung von Links im Verzeichnis der Internetquellen wurde entfernt, da es ein solches eigenständiges Verzeichnis nicht mehr gibt. 
\end{itemize}

\item \verb|text_mit_zitaten.tex|
\begin{itemize}
\item Es wird nun erläutert, wie zwei Quellenangaben unmittelbar nebeneinander dargestellt werden können.
\item Erklärungen, die von mehreren Quellenverzeichnissen ausgegangen sind, wurden entfernt.
\end{itemize}

\item \verb|literatur-datenbank.bib|
\begin{itemize}
\item Gespräch wurde entfernt, da dieses nicht mehr im Quellenverzeichnis aufgeführt werden soll.
\end{itemize}

\end{itemize}